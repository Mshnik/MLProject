

% Use the following line _only_ if you're still using LaTeX 2.09.
%\documentstyle[icml2012,epsf,natbib]{article}
% If you rely on Latex2e packages, like most moden people use this:
\documentclass{article}

% For figures
\usepackage{graphicx} % more modern
%\usepackage{epsfig} % less modern
\usepackage{subfigure} 

% For citations
\usepackage{natbib}

% For algorithms
\usepackage{algorithm}
\usepackage{algorithmic}

% As of 2011, we use the hyperref package to produce hyperlinks in the
% resulting PDF.  If this breaks your system, please commend out the
% following usepackage line and replace \usepackage{icml2012} with
% \usepackage[nohyperref]{icml2012} above.
\usepackage{hyperref}

% Packages hyperref and algorithmic misbehave sometimes.  We can fix
% this with the following command.
\newcommand{\theHalgorithm}{\arabic{algorithm}}

% Employ the following version of the ``usepackage'' statement for
% submitting the draft version of the paper for review.  This will set
% the note in the first column to ``Under review.  Do not distribute.''
\usepackage{icml2012} 
% Employ this version of the ``usepackage'' statement after the paper has
% been accepted, when creating the final version.  This will set the
% note in the first column to ``Appearing in''
% \usepackage[accepted]{icml2012}


% The \icmltitle you define below is probably too long as a header.
% Therefore, a short form for the running title is supplied here:
\icmltitlerunning{Submission and Formatting Instructions for ICML 2012}

\begin{document} 

\twocolumn[
\icmltitle{CS 4780 Final Project Propsal}



% You may provide any keywords that you 
% find helpful for describing your paper; these are used to populate 
% the "keywords" metadata in the PDF but will not be shown in the document
\icmlkeywords{boring formatting information, machine learning, ICML}

\vskip 0.3in
]

\section{Team}
This project will be completed by a team of four students.

\section{Motivation}
DonorsChoose is a crowd funding website that helps public school teachers request and receive funding. According to DonorsChoose, 70 percent of campaigns are successfully funded; this is a healthy margin, but could be vastly improved. It would be valuable for teachers to learn what factors may affect the success of their campaign.  

Extensive research about Kickstarter and other similar crowd-funding sources has already been done to investigate causes of successful commercial campaigns. It would be interesting to compare and contrast techniques for creating a successful crowd-funding campaign in the commercial realm to the philanthropic realm.

\section{Problem Statement}
Analyze dataset pertaining to donor excitement for academic projects. 

One goal of this project is to attempt determine what factors have the greatest influence on if a project will be fully funded.  Some factors we are interested in investigating is characteristics of the project such as the location of the school, the poverty level, the grade level, subject matter, and date posted.  We are also interested in looking at how characteristics of donations affect likelihood of success.

DonorChoose enables various promotions such as having a corporation match donations.  We would like to investigate if these promotions affect the number or size of donation as well as if they affect the likelihood of a project being funded.


 


Dataset includes campaign description essays; analyze text samples to determine if certain keywords or other features increase likelihood of funding.
\section{Approach}


\section{Resources}

Source is a closed Kaggle dataset, located here: https://www.kaggle.com/c/kdd-cup-2014-predicting-excitement-at-donors-choose/data

\section{Schedule}



24th October: Peer reviews for project proposal due (via CMT).
11th November: Submit progress report (via CMT).
04th December: Poster presentation. Submit poster (via CMT).
05th December: Peer reviews for posters due (via CMT).
10th December: Final project report (and code) due (via CMT).
15th December: Peer reviews for final project reports due (via CMT).
16th December: Author Feedback on reviews for final project reports due (via CMT).



\end{document} 
