

% Use the following line _only_ if you're still using LaTeX 2.09.
%\documentstyle[icml2012,epsf,natbib]{article}
% If you rely on Latex2e packages, like most moden people use this:
\documentclass{article}

% For figures
\usepackage{graphicx} % more modern
%\usepackage{epsfig} % less modern
\usepackage{subfigure} 

% For citations
\usepackage{natbib}

% For algorithms
\usepackage{algorithm}
\usepackage{algorithmic}

% As of 2011, we use the hyperref package to produce hyperlinks in the
% resulting PDF.  If this breaks your system, please commend out the
% following usepackage line and replace \usepackage{icml2012} with
% \usepackage[nohyperref]{icml2012} above.
\usepackage{hyperref}

% Packages hyperref and algorithmic misbehave sometimes.  We can fix
% this with the following command.
\newcommand{\theHalgorithm}{\arabic{algorithm}}

% Employ the following version of the ``usepackage'' statement for
% submitting the draft version of the paper for review.  This will set
% the note in the first column to ``Under review.  Do not distribute.''
\usepackage{icml2012} 
% Employ this version of the ``usepackage'' statement after the paper has
% been accepted, when creating the final version.  This will set the
% note in the first column to ``Appearing in''
% \usepackage[accepted]{icml2012}


% The \icmltitle you define below is probably too long as a header.
% Therefore, a short form for the running title is supplied here:
\icmltitlerunning{Submission and Formatting Instructions for ICML 2012}

\begin{document} 

\twocolumn[
\icmltitle{CS 4780 Final Project Propsal}



% You may provide any keywords that you 
% find helpful for describing your paper; these are used to populate 
% the "keywords" metadata in the PDF but will not be shown in the document
\icmlkeywords{boring formatting information, machine learning, ICML}

\vskip 0.3in
]

\section{Team}
This project will be completed by a team of four students.

\section{Motivation}
DonorsChoose is a crowd funding website that helps public school teachers request and receive funding. 70 percent of campaigns on DonorsChoose are successfully funded. While this is a healthy margin, it could be vastly improved. It would be valuable for teachers to learn what factors may affect the success of their campaign.  

Extensive research about Kickstarter and other similar crowd-funding sources has already been done to investigate causes of successful commercial campaigns. It would be interesting to compare and contrast techniques for creating a successful crowd-funding campaign in the commercial realm to the philanthropic realm.

\section{Problem Statement}
The main goal of this project is to determine what factors have the greatest influence on if a project will be fully funded.  in particular, we are interested in investigating whether characteristics of a project such as the location of the school, the poverty level, the grade level, or area of study (such as english versus chemistry) affect the likelihood of funding.  

We are also interested in looking at how characteristics of already pledged donations affect likelihood of future donations, and thus success of projects in a time-series framework.

DonorChoose enables various promotions such as having a corporation match donations.  We would like to investigate if these promotions affect the number or size of donation as well as if they affect the likelihood of a project being funded.

We also have access to project description essays, written by the creating teachers. We will analyze text samples to determine if certain keywords or other text features increase likelihood of funding for a given project.
\section{Approach}
The problems we are interested in solving are binary classification problems.  If we can determine how to classify the projects then we can investigate which characteristics are more influential.
We will use approaches learned in class for model selection 

We will use existing linear classification software.

\section{Resources}
The full dataset for this project is provided publicly by Kaggle. It is available here: https://www.kaggle.com/c/kdd-cup-2014-predicting-excitement-at-donors-choose/data

-something about the linear classifier software we plan on using...-

File reading and other custom code written for this project is written in Scala, a publicly accessible and OS-agnostic language.
\section{Schedule}

\begin{itemize}
\item 24th October: Peer reviews for project proposal due (via CMT).
\item 11th November: Submit progress report (via CMT).
\item 04th December: Poster presentation. Submit poster (via CMT).
\item 05th December: Peer reviews for posters due (via CMT).
\item 10th December: Final project report (and code) due (via CMT).
\item 15th December: Peer reviews for final project reports due (via CMT).
\item 16th December: Author Feedback on reviews for final project reports due (via CMT).
\end{itemize}

Tasks:
import data to database 
try stuff
determine way to compare results 
find different models 
compare said models
start poster and report
 


\end{document} 
